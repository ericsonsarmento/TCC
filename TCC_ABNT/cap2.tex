\chapter{Panorama Geral de Irrigação}
\label{CAP2}



\section{Conjuntura Mundial}\label{Sub:equa}

A irrigação, utilizada como técnica primordial para os cultivos em áreas com deficit hídrico elevado tem anualmente expandido sua área global, dados do FAOSTAT (2016) mostram que a área irrigada no mundo no ano de 2010 estava na casa dos 320 milhões de hectares, tendo um aumento de 5 milhões de hectares até o ano de 2013.
A mesma instituição afirma que nos últimos 10 anos o país teve um crescimento de 800 mil hectares irrigados, com acréscimo média de 200 mil por ano de 2006 até 2010, estabilizando em 5400 mil hectares até 2013.  

A Agencia Nacional de Água (ANA), órgão que monitora os recursos hídricos do país, informou em 2015 que a demanda conjuntiva de água chegou a 2.275 m cubico/s, tendo como maior contribuinte desse índice o setor de irrigação, detentor da parcela de 55 (porcento) sendo que o segundo maior consumidor de água é o abastecimento humano urbano com apenas 22 (porcento), contudo .

A área brasileira irrigada no ano de 2014 foi estimada em 6,11 milhões de hectares  ou 21 (porcento) do potencial nacional, que corresponde a 29,6 milhões de hectares, contudo, observa-se expressivo aumento da agricultura irrigada no Brasil, crescendo sempre a taxas superiores às do crescimento da área plantada total.

Investimentos em irrigação resultam em aumento substancial da produtividade e do valor da produção agrícola, diminuindo a necessidade de expansão em áreas ocupadas por outros usos e coberturas (pastagens ou matas nativas, por exemplo). Aplicando boas práticas de manejo do solo e da água, irrigantes alcançam efciências de uso dos recursos hídricos superiores a 90 (porcento). \cite{ana2015}.

\section{Uso Indevido da Água na Irrigação}
O crescimento populacional tem gerado demanda a cada ano por água, em consequência, a disponibilidade desse recurso vem tornando-se menor gradativamente, e o reflexo vem sendo observado não só no na escassez do elemento em si como na expansão das fronteiras agrícolas e à degradação do meio ambiente. Sendo a água um recurso indispensável à vida, é de fundamental importância a discussão das relações entre o homem e a água, uma vez que a sobrevivência das gerações futuras depende diretamente das decisões que hoje estão sendo tomadas \cite{lima1999uso}.

A escolha do sistema mais adequado depende de diversos fatores, \cite{de2006avaliaccao} enumera a topografia e o tipo de solo, a fonte de água (localização, vazão, qualidade), o sistema de plantio e o custo do equipamento e de operação. Contudo, no mesmo trabalho afirma que os sistemas que inicialmente tiveram um inadequado dimensionamento hidráulico, tiveram sua uniformidade de aplicação deficitária, causando decréscimo na produtividade devido ao uso irracional da água.


\section{Efeitos da Irrigação na Produção}

Em comparação com áreas não irrigadas a produção por hectare de culturas sob regime de irrigação, demonstra acrécimos em diversas áreas de cultivo. \cite{sanches2013girassol} desenvolvendo trabalhos com girassol, obteve altas significativas em áreas irrigadas, alcançado taxas de 62 (porcento) a mais que em áreas sem regime de irrigação. 

No cenário da pecuária, os estudos de \cite{sanches2013tifton} com capim tifton 85   sobresemeado com aveia, demonstraram índices mais elevados de matéria seca em kg/ha a partir do segundo siclo de pastejo. Os valores alcançaram a faixa de 82 (porcento) a mais de matéria seca nas parcelas irrigadas. O nível de proteína bruta também verificado apresentou acréscimos significativos.

Mesmo sendo a técnica de produção agrícola com utilização de um volume de água demasiadamente grande, essa é uma ação necessária, pois a aplicação de água nas culturas aumenta a eficiência de uso de outros insumos, como fertilizantes, por exemplo, garante a produção na entressafra em regiões áridas ou de regime pluviométrico inconstante, além de oferecer segurança durante os veranicos \cite{de2008desenvolvimento}.
